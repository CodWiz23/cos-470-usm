%+++++++++++++++++++++++++++++++++++++++++++++++++++++++++++++++
% SUMMARY    : Title
%            : University of Southern Maine 
%            : @james.quinlan
%+++++++++++++++++++++++++++++++++++++++++++++++++++++++++++++++

\section*{Objectives}
\begin{outline}
    \1 Objective 1
    \1 Objective 2
    \1 Objective 3
\end{outline}


\rule[0.0051in]{\textwidth}{0.00025in}
% ----------------------------------------------------------------


\section{Mathematics}
Inline mathematics is enclosed with dollar signs, \$math\$.  For example, $x^2 = 2 \Rightarrow x = \sqrt{2}$.  
You can center mathematics with double dollar signs (or backslash square brackets, \verb@\[ math \]@.
\[
x^2 = 2 \Rightarrow x = \sqrt{2}
\]

Use an equation or align environment for centered numbered equations or a chain of equations.
If equation numbering is not needed (i.e., not referenced), the \verb@align*@ can be used instead.
The ``align" environment is preferred in general. 
%
\begin{align}
x^2 - 5x + 6 	&= (x - 2)(x - 3) \\
		&= x = 2 \quad \lor \quad x = 3
\end{align}
%

\section{Table and Figure}

%% Table
\begin{table}[h!]
    \centering
    \begin{tabular}{c c}
     \toprule
    Header 1 &   Header 1 \\
      \midrule
  	A & B\\
	C & D\\
	\bottomrule
    \end{tabular}
    \caption{Describe}
    \label{tab:abc}
\end{table}
%
%
\begin{figure}[htbp] %  figure placement: here, top, bottom, or page
   \centering
   \includegraphics[width=2in]{figures/example} 
   \caption{example caption}
   \label{fig:example}
\end{figure}

% Side-by-side subfigures 
\begin{figure}
	\begin{subfigure}[h]{0.4\linewidth}
		\includegraphics[width=\linewidth]{figures/example}
		\caption{Image A}
	\end{subfigure}
		\hfill
	\begin{subfigure}[h]{0.4\linewidth}
		\includegraphics[width=\linewidth]{figures/example}
		\caption{Image B}
	\end{subfigure}%
\caption{This is a figure with two subfigures}
\end{figure}






\section{Algorithm and Code}
% Algorithm
\begin{algorithm}[h]
\caption{Round and replace}\label{alg:21}
\begin{algorithmic}[1] % Line numbers
	\Require $x$
	\Ensure $y$

	\State $y = f(x)$
	\If{x > 1}
		\State $y = 2$
	\EndIf

\end{algorithmic}
\end{algorithm}


Code can be rendered with the verbatim environment or in the listings package.

\begin{verbatim}
if (1 > 0)
  print("hello world")
end
\end{verbatim}

\begin{lstlisting}[language=Octave]
if (1 > 0)
  print("hello world")
end
\end{lstlisting}
