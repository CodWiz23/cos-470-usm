\usepackage{epsfig}
\usepackage{amsfonts}
\usepackage{amssymb}
\usepackage{amstext}
\usepackage{amscd}
\usepackage{amsmath}
\usepackage{xspace}
\usepackage{theorem}
\usepackage{float}
\usepackage{mathabx, color} % square bullets in itemized list
%\usepackage{layout}% if you want to see the layout parameters
                     % and now use \layout command in the body
\usepackage{url}
\usepackage{pgfplots}
\pgfplotsset{width=7cm,compat=1.10}

\usepackage{soul}
% Frame package and color package - used for colored frame boxes  

\usepackage{marginnote}

% \usepackage{framed,color, mcode}
% \usepackage{mcode}
%\lstset{
%    literate={~} {$\sim$}{1}
%}


% Define colors
\definecolor{shadecolor}{rgb}{0.95,0.975,0.997}
\definecolor{shadecode}{rgb}{0.91,0.91,0.91}
\definecolor{red_1}{rgb}{1,0.8,0.8}
\definecolor{yellow_1}{rgb}{1,0.96,0.63}
\definecolor{orange}{rgb}{1,0.5,0}
\definecolor{appleGray}{rgb}{0.75,0.75,0.75}
\definecolor{lightGray}{rgb}{0.975,0.975,0.975}
\definecolor{borderGray}{rgb}{0.8,0.8,0.8}
\definecolor{orange}{cmyk}{0,0.5,1,0}


% This is the stuff for normal spacing
\makeatletter
 \setlength{\textwidth}{6.5in}
 \setlength{\oddsidemargin}{0in}
 \setlength{\evensidemargin}{0in}
 \setlength{\topmargin}{0.25in}
 \setlength{\textheight}{8.25in}
 \setlength{\headheight}{0pt}
 \setlength{\headsep}{0pt}
 \setlength{\marginparwidth}{59pt}

 \setlength{\parindent}{0pt}
 \setlength{\parskip}{5pt plus 1pt}
 \setlength{\theorempreskipamount}{5pt plus 1pt}
 \setlength{\theorempostskipamount}{0pt}
 \setlength{\abovedisplayskip}{8pt plus 3pt minus 6pt}
 \setlength{\intextsep}{15pt plus 3pt minus 6pt}
 
% square bullet (also use package above)
% \renewcommand{\labelitemi}{\color{orange}$\square$}
%\renewcommand{\labelitemi}{\includegraphics[]{blue_bullet.png}}

 \renewcommand{\section}{\@startsection{section}{1}{0mm}%
                                   {2ex plus -1ex minus -.2ex}%
                                   {1.3ex plus .2ex}%
                                   {\normalfont\Large\bfseries}}%
 \renewcommand{\subsection}{\@startsection{subsection}{2}{0mm}%
                                     {1ex plus -1ex minus -.2ex}%
                                     {1ex plus .2ex}%
                                     {\normalfont\large\bfseries}}%
 \renewcommand{\subsubsection}{\@startsection{subsubsection}{3}{0mm}%
                                     {1ex plus -1ex minus -.2ex}%
                                     {1ex plus .2ex}%
                                     {\normalfont\normalsize\bfseries}}
 \renewcommand\paragraph{\@startsection{paragraph}{4}{0mm}%
                                    {1ex \@plus1ex \@minus.2ex}%
                                    {-1em}%
                                    {\normalfont\normalsize\bfseries}}
 \renewcommand\subparagraph{\@startsection{subparagraph}{5}{\parindent}%
                                       {2.0ex \@plus1ex \@minus .2ex}%
                                       {-1em}%
                                      {\normalfont\normalsize\bfseries}}
\makeatother

\newcounter{thelecture}

\newenvironment{proof}{{\bf Proof:  }}{\hfill\rule{2mm}{2mm}}
\newenvironment{proofof}[1]{{\bf Proof of #1:  }}{\hfill\rule{2mm}{2mm}}
\newenvironment{proofofnobox}[1]{{\bf#1:  }}{}
\newenvironment{example}{{\bf Example:  }}{\hfill\rule{2mm}{2mm}}

%\renewcommand{\theequation}{\thesection.\arabic{equation}}
%\renewcommand{\thefigure}{\thesection.\arabic{figure}}

\newtheorem{fact}{Fact}
\newtheorem{lemma}[fact]{Lemma}
\newtheorem{theorem}[fact]{Theorem}
\newtheorem{definition}[fact]{Definition}
\newtheorem{corollary}[fact]{Corollary}
\newtheorem{proposition}[fact]{Proposition}
\newtheorem{claim}[fact]{Claim}
\newtheorem{exercise}[fact]{Exercise}

% math notation
\newcommand{\R}{\ensuremath{\mathbb R}}
\newcommand{\Z}{\ensuremath{\mathbb Z}}
\newcommand{\N}{\ensuremath{\mathbb N}}
\newcommand{\B}{\ensuremath{\mathbb B}}
\newcommand{\F}{\ensuremath{\mathcal F}}
\newcommand{\SymGrp}{\ensuremath{\mathfrak S}}
\newcommand{\prob}[1]{\ensuremath{\text{{\bf Pr}$\left[#1\right]$}}}
\newcommand{\expct}[1]{\ensuremath{\text{{\bf E}$\left[#1\right]$}}}
\newcommand{\size}[1]{\ensuremath{\left|#1\right|}}
\newcommand{\ceil}[1]{\ensuremath{\left\lceil#1\right\rceil}}
\newcommand{\floor}[1]{\ensuremath{\left\lfloor#1\right\rfloor}}
\newcommand{\ang}[1]{\ensuremath{\langle{#1}\rangle}}
\newcommand{\poly}{\operatorname{poly}}
\newcommand{\polylog}{\operatorname{polylog}}

% anupam's abbreviations
\newcommand{\e}{\epsilon}
%\newcommand{\half}{\ensuremath{\frac{1}{2}}}
\newcommand{\junk}[1]{}
\newcommand{\sse}{\subseteq}
\newcommand{\union}{\cup}
\newcommand{\meet}{\wedge}
\newcommand{\dist}[1]{\|{#1}\|_{\text{dist}}}
\newcommand{\hooklongrightarrow}{\lhook\joinrel\longrightarrow}
\newcommand{\embeds}[1]{\;\lhook\joinrel\xrightarrow{#1}\;}
\newcommand{\mnote}[1]{\normalmarginpar \marginpar{\tiny #1}}


\usepackage{outlines}
\usepackage{enumitem}
\setenumerate[1]{label=\arabic*.}
\setenumerate[2]{label=(\alph*).}
\setenumerate[3]{label=\roman*.}
% \setenumerate[4]{label=\alph*.}

\usepackage{booktabs}
\usepackage{tcolorbox, graphicx}
\usepackage{xcolor,colortbl}
\usepackage{framed,color}
\definecolor{excel}{rgb}{0.90,0.975,0.997}


% PACKAGES
% -----------------
\usepackage{framed}  % https://latexcolor.com
\usepackage{xcolor}
\usepackage{tcolorbox, graphicx}
\usepackage{colortbl}
\usepackage{textcomp}
\usepackage{listings}  

% Colors
% -----------------
\definecolor{antiquewhite}{rgb}{0.96, 0.96, 0.96} % {0.98, 0.92, 0.84}
\definecolor{appleGray}{rgb}{0.75,0.75,0.75}
\definecolor{lightGray}{rgb}{0.975,0.975,0.975}
\definecolor{relhead}{rgb}{0.70,0.80,0.90}
\definecolor{borderGray}{rgb}{0.8,0.8,0.8}
\definecolor{nearwhite}{rgb}{0.985,0.985,0.985}
\definecolor{mygray}{rgb}{0.20,0.20,0.20}
\definecolor{mymauve}{rgb}{0.58,0,0.82}
\definecolor{excel}{rgb}{0.94, 0.94, 0.94}
%\definecolor{anti-flashwhite}{rgb}{0.95, 0.95, 0.96}
%\definecolor{magnolia}{rgb}{0.97, 0.96, 1.0}
%\definecolor{shadecolor}{rgb}{0.95,0.975,0.997}
%\definecolor{shadecode}{rgb}{0.91,0.91,0.91}
%\definecolor{red_1}{rgb}{1,0.8,0.8}
%\definecolor{yellow_1}{rgb}{1,0.96,0.63}
%\definecolor{orange}{rgb}{1,0.5,0}
%\definecolor{gray}{rgb}{0.975,0.975,0.975}
%\definecolor{supergray}{cmyk}{0,0,0.04,0}
%\definecolor{stainlessSteel}{cmyk}{0,0,0.02,0.12}
\definecolor{mygreen}{rgb}{0,0.6,0}


 \lstset{ 
  backgroundcolor=\color{white},	% background color; e.g., nearwhite you must add \usepackage{color} or \usepackage{xcolor};
  basicstyle=\footnotesize\ttfamily,	% the size of the fonts that are used for the code
  breakatwhitespace=false,		% sets if automatic breaks should only happen at whitespace
  breaklines=true,			% sets automatic line breaking
  framextopmargin=5pt,
  framexleftmargin=5pt, 
  framexbottommargin=5pt,
  framexrightmargin=0pt,
  framesep=0pt,
  captionpos=b,				% sets the caption-position to bottom
  commentstyle=\color{mygreen},		% comment style
  morecomment=[s]{/*}{*/},
  deletekeywords={...},			% if want to delete keywords from the given language
  escapeinside={\%*}{*)},		% if you want to add LaTeX within your code
  extendedchars=true,			% use non-ASCII chars; 8-bits encodings only, not work with UTF-8
  frame=single,				% adds a frame around the code
  keepspaces=false,			% keeps spaces in text, useful indentation of code (possibly needs columns=flexible)
  keywordstyle=\color{blue},		% keyword style blue
  language=python,			% the language of the code
  morekeywords={ORDER, OUT, CALL},	% if you want to add more keywords to the set
  numbers=none,				% where are line-numbers; possible values are (none,left,right)
  numbersep=0pt,			% how far the line-numbers are from the code
  numberstyle=\tiny\color{mygray},	% the style that is used for the line-numbers
  rulecolor=\color{appleGray},		% if not set, frame-color may be changed on line-breaks within not-black text 
  showspaces=false,			% show spaces everywhere adding particular underscores; it overrides 'showstringspaces'
  showstringspaces=false,		% underline spaces within strings only
  showtabs=false,			% show tabs within strings adding particular underscores
  stepnumber=2,				% the step between two line-numbers. If it's 1, each line will be numbered
  stringstyle=\color{mymauve},		% string literal style
  tabsize=4,				% sets default tabsize to 2 spaces
  title=\lstname,			% show filename included with \lstinputlisting; try caption instead of title
  upquote=true,				% Straight quotes
  belowcaptionskip=0em,
  belowskip=0em
}
 




% Trig stuff
\DeclareMathOperator{\arcsec}{arcsec}
\DeclareMathOperator{\arccot}{arccot}
\DeclareMathOperator{\arccsc}{arccsc}
