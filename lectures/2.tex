%+++++++++++++++++++++++++++++++++++++++++++++++++++++++++++++++
% SUMMARY    : Norms, Inner Product, and Projection
%            : University of Southern Maine 
%            : @james.quinlan
%	     : James Tedder - Lecture 2
%+++++++++++++++++++++++++++++++++++++++++++++++++++++++++++++++

\section*{Objectives}
\begin{outline}
    \1 Norms
    \1 Inner Product
    \1 Projection
\end{outline}

\rule[0.0051in]{\textwidth}{0.00025in}
% ----------------------------------------------------------------
\section{Norms}

A Norm is a function that takes a vector and maps it to a real number such that:

\begin{enumerate}
	\item norms are greater than or equal to 0
	\item norms are equal to 0 only if x=0
	\item $||Cx|| = |C| \cdot ||x|| \text{ where } C \in \R$
	\item $||x+y|| \leq ||x|| + ||y||$
\end{enumerate}

An example of using these rules is when determining whether something is a norm is with the Hamming Distance. Which property doesn't the hamming distance satisfy? Hamming distance is a sum of different parts. It doesn't fit rule 3 because the result is not multiplied  by C if you multiply the vector by 3. With a hamming distance remains the same.

There are several types of norms and we differentiate them with subscripts. The first set of norms are the Minkowski norms or p norms and they are denoted as such.

\centering
1-norm $= ||x||_1$

2-norm $= ||x||_2$

3-norm $= ||x||_3$

p-norm $= ||x||_p$

\raggedright

These norms are calculated differently depending on the subscripts. The equations are shown below. In the first two cases p is the dimensionality of x and in the third dimension n is the dimensionality of x.

\centering
$$||x||_1 = \sum_{i=1}^{p}|x_i|$$

$$||x||_2 = \sqrt{\sum_{x=1}^{p}x_i^2}$$

$||x||_p = \sqrt{\sum_{i=1}^{n}|x_i|^p}$

\raggedright

Chebyshev's norm also referred to as the $\infty$ norm or the max norm is denoted as $||x||_\infty$. This norm finds the component of the vector that is the farthest from 0 and returns that. The technical definition is as follows.

$$||x||_\infty = max(|x_1|, |x_2|, |x_3|, ... |x_p|)$$

All norms are equivalent but not equal. If $||x||_1$ is larger than $||y||_1$ then $||x||_p$ is larger than $||y||_p$.

% ----------------------------------------------------------------
\section{Inner Product}

The inner product is $\langle\cdot,\cdot\rangle: V \times V \rightarrow\R$ such that:

\begin{enumerate}
	\item $\langle x,x \rangle \geq 0$
	\item $\langle x,x \rangle = 0 \text{iff} x=0$
	\item $\langle x,y \rangle = \langle y,x \rangle$
	\item $\langle \alpha x+ \beta y,z \rangle = \alpha \langle x,z \rangle + \beta \langle y,z 			\rangle$,				$\alpha,\beta \in \R$,
			$x,y,z \in V$
\end{enumerate}

This is the basis of the neural networks. A few more aspects of inner product are:

$\langle x,y \rangle = x^T y = \sum_{i=1}^{P}x_i y_i$

$||x|| = \sqrt{\langle x,x \rangle}$


% ----------------------------------------------------------------
\section{Projection}

Projection is used in principle component analysis. We started by looking at the problem to show a situation in which projection might be necessary.

$Ax=b$

A is a 2d matrix, b is a known vector, and x is an unknown vector.

$\begin{bmatrix}
1 & 2 & 3\\
4 & 5 & 6\\
7 & 8 & 9
\end{bmatrix}
\begin{pmatrix}
x_1\\
x_2\\
x_3
\end{pmatrix}
=
\begin{pmatrix}
10\\
11\\
12
\end{pmatrix}$

This problem has us trying to find the vector x. This problem may have no perfect solution. This is the sort of problem where we might use projection to solve it because projection is the best solution we can come up with.

He then showed another problem.

\begin{tikzpicture}
	\draw[thick,->] (0,0) -- (4.5,0) node[anchor=north west] {x axis};
	\draw[thick,->] (0,0) -- (0,4.5) node[anchor=south east] {y axis};
	\draw[thick,->] (0,0) -- (4,1) node[anchor=south east] {a};
	\draw[thick,->] (0,0) -- (2.5,2) node[anchor=south east] {b};
\end{tikzpicture}

The goal was to project b down onto a. This is necessary because there is no way to turn a into b through multiplication. Multiples just stretches vector a. There are a few terms we'll need to know before we can do this.

$p =$ projection which is equivalent to $b_\|$ which is parallel to a. It is also equivalent to a time some constant as it is just a resizing of a or $b_\| = ca = p$. This is the vector added below.

\begin{tikzpicture}
	\draw[thick,->] (0,0) -- (4.5,0) node[anchor=north west] {x axis};
	\draw[thick,->] (0,0) -- (0,4.5) node[anchor=south east] {y axis};
	\draw[thick,->] (0,0) -- (4,1) node[anchor=south east] {a};
	\draw[thick,->] (0,0) -- (2.5,2) node[anchor=south east] {b};
	\draw[thick,->] (0,0) -- (3,0.75) node[anchor=north east] {$b_\|$};
\end{tikzpicture}

$b_\perp$ is perpendicular to a. It is also know as the residual and is equal to b minus a times some constant c or $b_\perp = b-ca$. This is the vector added below.

\begin{tikzpicture}
	\draw[thick,->] (0,0) -- (4.5,0) node[anchor=north west] {x axis};
	\draw[thick,->] (0,0) -- (0,4.5) node[anchor=south east] {y axis};
	\draw[thick,->] (0,0) -- (4,1) node[anchor=south east] {a};
	\draw[thick,->] (0,0) -- (2.5,2) node[anchor=south east] {b};
	\draw[thick,->] (0,0) -- (3,0.75) node[anchor=north west] {$b_\|$};
	\draw[thick,->] (2.5,2) -- (3,0.75) node[anchor=south east] {$b_\perp$};
\end{tikzpicture}

It is also important to note that $b = b_\| + b_\perp$ and that $\langle b_\| , b_\perp \rangle = 0$ we use all of these definitions and the inner product to solve it.

\begin{align*}
	0 &= \langle ca, b-ca \rangle \\
	0 &= (ca)^T(b-ca) \\
	0 &= ca^T (b-ca) \\
	0 &= ca^Tb-c^2a^Tb \\
	ca^Ta  &= a^Tb \\
	c &= \frac{a^Tb}{a^Ta} \\
	c &= \frac{\langle a,b \rangle}{\langle a,a \rangle} \\
	ca &=  (\frac{a^Tb}{a^Ta})a 
\end{align*}

All of this allows you to project b onto a.

